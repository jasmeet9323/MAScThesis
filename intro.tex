%% The following is a directive for TeXShop to indicate the main file
%%!TEX root = diss.tex

\chapter{Introduction}
\label{ch:Introduction}

\begin{epigraph}
    \emph{If I have seen farther it is by standing on the shoulders of
    Giants.} ---~Sir Isaac Newton (1855)
\end{epigraph}

Computational Fluid Simulations (CFD) is a field of study where scientists and engineers architect new ways to numerically solve fluid flow equations. Before the advent of computers, numerical solutions of differential equations was done by hand. This lead to a great deal of work in the direction of creating faster algorithms to solve differential equations. An example is the development of the Fast Fourier Transform (FFT) by Cornelius Lancoz to increase the computation speed of Discrete Fourier Transform (DFT). However, since the development and advancement of computers, engineers had a significant amount of compute power to work with. This lead to the development of highly accurate methods (as compared to before) to simulate flow over various objects. These simulations have since gotten bigger and better, typically including millions of degrees of freedom, even starting to touch a billion in regular industry use.

The equations which govern the conservation of mass, momentum and energy of a moving fluid also called Navier-Stokes equations are solved in the given domain to simulate fluid flow in that domain. So as to numerically solve these equations, we need a discretization of the given domain. This discrete basis required to solve the Navier-Stokes equations is called a mesh. Simply put, a mesh is a collection of points, lines and cells that together construct the space around a body in a fluid flow.



