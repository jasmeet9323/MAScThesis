\chapter{Example Meshes}

In the first chapter, we gave a brief introduction to mesh generation. We discussed about structured, unstructured, simplical and non-simplical meshes and included a brief literature review of the research papers in the field of mesh generation. There, we stressed on the importance of generating anisotropic meshes or boundary layer meshes when solving for boundary layer phenomena and motivated the problem of generating stretched quad-dominant surface meshes to serve as the starting point of complete three dimensional anisotropic volume mesh generation.

We started discussing about the method we developed to generate such anisotropic quad-dominant meshes in the second chapter. Here, we discussed about the surface import, point placement, local reconnection and front recovery subroutines which are an essential part of the mesh generation process. In the third chapter, we discussed the subroutines used to advance several layers of the mesh and close off the marching layers so as to form a complete surface mesh. Controlling the aspect ratio on the advancing front, combining triangular elements to quadrilateral ones, mesh smoothing and collision handling were discussed.

In this chapter, we go over some of the example meshes generated by using EDAMSurf or Entire Domain Advancing Layer Mesh - Surface.

\section{NACA0018}

We take the NACA0018 airfoil 2D profile and extrude it in 3D. This gives us a three dimensional airfoil as shown in the figure.