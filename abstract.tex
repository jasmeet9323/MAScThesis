%% The following is a directive for TeXShop to indicate the main file
%%!TEX root = diss.tex

\chapter{Abstract}

Use of unstructured meshes in the simulation of a computational field to solve for a real world problem is ubiquitous. Specially, solving fluid flow over bodies like an airplane or a turbine computationally requires a well discretized domain, or a mesh around the surfaces of these bodies. In Computational Fluid Dynamic (CFD) simulations over these surfaces, the flow at the viscous-boundary layer of the surface is very important as the gradients in the normal direction of the flow are sharp and are orders of magnitude higher than the gradients in the tangential direction of the flow. Hence, resolving the flow field in the boundary layer is vital for accurate simulation results.

A plethora of 3D boundary layer mesh generation techniques start off from a discretization of the surface. A majority of these techniques either use surface inflation or iterative point placement normal to the surface to generate the advancing layer 3D mesh. Generating boundary layer meshes in 3D depends on the quality of the underlying surface discretization. We introduce a technique to generate advancing layer surface meshes which would improve the mesh generation pipline for 3D mesh generation. The technique takes an input triangulation of the surface, which is fairly easy to get, even for complex geometries. Surface segments are identified and these segmentsare meshed independently using a advancing-layer methodology. For each surface segment, a mesh is generated by advancing layers from the identified boundaries to the surface interior while deforming the existing triangulation. As the mesh-generation technique introduced here produces a closed-mesh, we get a valid mesh at each iteration of advancing layer.

The method introduced to generate advancing layer meshes 

A Viscous-boundary layer mesh generation technique based on advancing layer 

Fluid flow over an object is ubiquitous in real-world problems. Computational Fluid Dynamics (CFD) simulations try to reproduce the physics involved in such problems without the need of doing experiments. These simulations utilize the discretized domain around the object, also called a mesh to proceed with the solution. 

% Consider placing version information if you circulate multiple drafts
%\vfill
%\begin{center}
%\begin{sf}
%\fbox{Revision: \today}
%\end{sf}
%\end{center}
