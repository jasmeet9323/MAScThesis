\chapter{Summary}

\section{Concluding Comments}

In the field of Computational Fluid Dynamics, mesh generation and adaptation still takes considerable effort. It requires the mesh developer to have significant knowledge about the application of the mesh. This is more true in applications entailing boundary layer flows which requires a mesh with anisotropic or stretched elements. Creating valid anisotropic meshes for a diverse set of geometrical bodies is thus, a tedious and difficult task. Hence, some sort of automation in the development of anisotropic meshes will simplify the process of mesh generation for engineers and scientists. This thesis may represents one such step towards generating fully automatic anisotropic three-dimensional meshes.

Most of the three-dimensional anisotropic meshes start from the discretization of the surface. This discretization is generally an isotropic triangulation of the surface. In this thesis, we argue that an isotropic triangulation of the surface cannot be a foundation for completely anisotropic three-dimensional mesh. Hence, the mesh generation algorithm presented, called Entire Domain Advancing Layer Mesh Generation --- Surface (EDAMSurf), produces an anisotropic surface mesh from a given input triangulation of a solid body. The mesh generation algorithm initializes advancing fronts from the boundary curves of the surface patches of the input triangulation. The fronts advance layer-by-layer with the given growth ratio and sweep the surface of the geometry. Special subroutines are implemented to handle mesh smoothing, combining triangles to quadrilaterals, and to handle advancing front collisions.

EDAMSurf generates quad-dominant meshes with a few triangular elements. This is done to achieve low vertex connectivity in the mesh. Additionally, having non-simplicial elements in the mesh may lead to cancellation of local truncation errors while solving for fluid flows (See \ref{sec-simplicial}). The meshes generated by EDAMSurf generally have more than 95\% of the mesh elements as quadrilateral elements. Triangular elements are included in the mesh whenever there is a need to deal with complex corners or advancing layer collisions. Four example meshes that demonstrate the capabilities of the application are shown. The distribution of the interior angles of the mesh elements in the meshes generated by EDAMSurf show a peak at 90$^\circ$. This is because majority of mesh elements are rectangular in shape. Generally, mesh elements have more than 90\% of the angles in between 45$^\circ$ and 135$^\circ$. More triangular elements in the mesh give us more flattened angle distribution for the mesh elements.

EDAMSurf is shown to handle sharp concave corners in the input geometries well. It successfully retrieves advancing front definition while front collisions happen at the concave corners. The definition of the boundary curves is advanced several layers onto the surface mesh. Also, EDAMSurf is able to handle geometries with diverse topologies and holes. Example cases are demonstrated where multiple advancing layers collapse in a very small region and the mesh generation algorithm is able to successfully continue marching with a valid front definition.

\section{Future Work}

Automatic anisotropic meshing of complex three-dimensional domains is still a tricky task. EDAMSurf provides a step towards it, but is in no way a perfect one size fits all solution. EDAMSurf is a part of the project GRUMMP (Generation and Refinement of Unstructured Mixed-Element Meshes in Parallel)\cite{ollivier2010grummp} which is in continuous development. Some steps that might be taken in the future for improving and/or extending the mesh generation scheme are -

\begin{enumerate}
	\item EDAM3D is an extension of EDAMSurf. Being developed at ANSLab, UBC, it aims to develop completely anisotropic three-dimensional volume meshes for fluid flow applications.
	\item EDAMSurf handles concave corners robustly. Marching directions are eliminated when vertices on the advancing front collide with each other. In the future, a subroutine might be added to the algorithm which adds multiple marching directions to the convex corners of the surface so as to improve mesh quality.
	\item Due to the restrictions of Common Geometry Module's Application Programming Interface, EDAMSurf could not use curvature information on the imported surface to modify the marching direction as well as extrusion lengths on the front. A subroutine which is able to achieve this might allow EDAMSurf to tackle highly curved geometries with poor input discretization.
	\item EDAMSurf generates meshes of sub surfaces independently. Hence, a performance speedup could be attained by parallelising the application.
\end{enumerate}








